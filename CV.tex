%-----------------------------------------------------------------------------------------------------------------------------------------------%
%	The MIT License (MIT)
%
%	Copyright (c) 2021 Jitin Nair
%
%	Permission is hereby granted, free of charge, to any person obtaining a copy
%	of this software and associated documentation files (the "Software"), to deal
%	in the Software without restriction, including without limitation the rights
%	to use, copy, modify, merge, publish, distribute, sublicense, and/or sell
%	copies of the Software, and to permit persons to whom the Software is
%	furnished to do so, subject to the following conditions:
%	
%	THE SOFTWARE IS PROVIDED "AS IS", WITHOUT WARRANTY OF ANY KIND, EXPRESS OR
%	IMPLIED, INCLUDING BUT NOT LIMITED TO THE WARRANTIES OF MERCHANTABILITY,
%	FITNESS FOR A PARTICULAR PURPOSE AND NONINFRINGEMENT. IN NO EVENT SHALL THE
%	AUTHORS OR COPYRIGHT HOLDERS BE LIABLE FOR ANY CLAIM, DAMAGES OR OTHER
%	LIABILITY, WHETHER IN AN ACTION OF CONTRACT, TORT OR OTHERWISE, ARISING FROM,
%	OUT OF OR IN CONNECTION WITH THE SOFTWARE OR THE USE OR OTHER DEALINGS IN
%	THE SOFTWARE.
%	
%
%-----------------------------------------------------------------------------------------------------------------------------------------------%

%----------------------------------------------------------------------------------------
%	DOCUMENT DEFINITION
%----------------------------------------------------------------------------------------

% article class because we want to fully customize the page and not use a cv template
\documentclass[a4paper,12pt]{article}

%----------------------------------------------------------------------------------------
%	FONT
%----------------------------------------------------------------------------------------

% % fontspec allows you to use TTF/OTF fonts directly
% \usepackage{fontspec}
% \defaultfontfeatures{Ligatures=TeX}

% % modified for ShareLaTeX use
% \setmainfont[
% SmallCapsFont = Fontin-SmallCaps.otf,
% BoldFont = Fontin-Bold.otf,
% ItalicFont = Fontin-Italic.otf
% ]
% {Fontin.otf}

%----------------------------------------------------------------------------------------
%	PACKAGES
%----------------------------------------------------------------------------------------
\usepackage{url}
\usepackage{parskip} 	

%other packages for formatting
\RequirePackage{color}
\RequirePackage{graphicx}
\usepackage[usenames,dvipsnames]{xcolor}
\usepackage[scale=0.9]{geometry}
\geometry{top=1cm}
\geometry{bottom=1cm}

%tabularx environment
\usepackage{tabularx}

%for lists within experienontinuing me section
\usepackage{enumitem}

% centered version of 'X' col. type
\newcolumntype{C}{>{\centering\arraybackslash}X} 

%to prevent spillover of tabular into next pages
\usepackage{supertabular}
\usepackage{tabularx}
\newlength{\fullcollw}
\setlength{\fullcollw}{0.47\textwidth}

%custom \section
\usepackage{titlesec}				
\usepackage{multicol}
\usepackage{multirow}
\usepackage{setspace}

%CV Sections inspired by: 
%http://stefano.italians.nl/archives/26
\titleformat{\section}{\Large\scshape\raggedright}{}{0em}{}[\titlerule]
\titlespacing{\section}{0pt}{10pt}{10pt}

%for publications
\usepackage[style=authoryear,sorting=ynt, maxbibnames=2]{biblatex}

%Setup hyperref package, and colours for links
\usepackage[unicode, draft=false]{hyperref}
\definecolor{linkcolour}{rgb}{0,0.2,0.6}
\hypersetup{colorlinks,breaklinks,urlcolor=linkcolour,linkcolor=linkcolour}
\addbibresource{citations.bib}
\setlength\bibitemsep{1em}

%for social icons
\usepackage{fontawesome5}

%debug page outer frames
%\usepackage{showframe}

%----------------------------------------------------------------------------------------
%	BEGIN DOCUMENT
%----------------------------------------------------------------------------------------
\begin{document}

% non-numbered pages
\pagestyle{empty} 

%----------------------------------------------------------------------------------------
%	TITLE
%----------------------------------------------------------------------------------------

% \begin{tabularx}{\linewidth}{ @{}X X@{} }
% \huge{Your Name}\vspace{2pt} & \hfill \emoji{incoming-envelope} email@email.com \\
% \raisebox{-0.05\height}\faGithub\ username \ | \
% \raisebox{-0.00\height}\faLinkedin\ username \ | \ \raisebox{-0.05\height}\faGlobe \ mysite.com  & \hfill \emoji{calling} number
% \end{tabularx}
% \vspace{-4em}
\begin{tabularx}{\linewidth}{@{} C @{}}
\Huge{Arjun Pitchanathan} \\[7.5pt]
\href{mailto:arjunpitchanathan@gmail.com}{\raisebox{-0.05\height}\faEnvelope \ arjunpitchanathan@gmail.com} \ $|$ \ 
Work Authorized in the UK (Global Talent Visa) % \ $|$ \ 
% \href{https://github.com/Superty}{\raisebox{-0.05\height}\faGithub\ superty} % \ $|$ \ 
% \href{https://linkedin.com/in/username}{\raisebox{-0.05\height}\faLinkedin\ username} \ $|$ \ 
% \href{https://mysite.com}{\raisebox{-0.05\height}\faGlobe \ mysite.com} \ $|$ \ 

% \href{tel:+000000000000}{\raisebox{-0.05\height}\faMobile \ +00.00.000.000} \\
\end{tabularx}

% \textbf{Dean's List.} Awarded the Dean's merit list for academic excellence twice. % for the Monsoon '18 and Monsoon '19 semesters.

% \cvitem{ACM ICPC 2018}{Member of team tesla\_protocol which came \href{https://icpc.baylor.edu/regionals/finder/ICPC-Gwalior-Pune-2018/standings}{\textbf{3rd} in the Asia Gwalior-Pune Onsite Round} and \href{https://icpc.baylor.edu/regionals/finder/AMP-2018/standings}{\textbf{5th} in the Asia Amritapuri Onsite Round}. The ACM ICPC is the most prestigious university-level programming competition. }

%\cvitem{Codeforces}{Peak rating: 2225. \textbf{Top 2000 in the world} out of 175000+ users (98.8 percentile); Peak ranking: \textbf{top 15 in India} out of 5000+ users. Handle: Superty. Codeforces is a website that regularly hosts algorithmic programming competitions. Rating is based on performance in these competitions.}

% \cvitem{Codeforces}{Peak rating: 2225. \textbf{Top 2000 in the world} out of 175000+ users (98.8 percentile); \textbf{top 15 in India} out of 5000+ users. Handle: Superty. Codeforces is a website that regularly hosts algorithmic programming competitions. Rating is based on performance in these competitions.}

%\cvitem{ACM ICPC 2015}{Member of team \underline{not\_interested} which came \underline{\href{https://icpc.baylor.edu/regionals/finder/amp-2015/standings}{32nd in the Asia-Amritapuri Onsite Round}}. We also came \underline{\href{https://icpc.baylor.edu/regionals/finder/asia-amritapuri-first-round-2015/standings}{36th in the Asia-Amritapuri Online Regionals}} out of 1500+ teams that participated and \underline{\href{https://icpc.baylor.edu/regionals/finder/chennai-india-2015/standings}{22nd in the Asia-Chennai Online Regionals}}, out of 1000+ teams. The ACM ICPC is the most prestigious programming competition at the college level. }

% \cvitem{CodeChef SnackDown 2016}{Member of team lamecoders which placed \textbf{6th among Indian teams} and was \textbf{ranked 24th internationally} in the final round. We were ranked 21st among Indian teams in the online elimination round out of 1000+ Indian teams.}


% \cventry{2019--2021 (expected)}{MS by Research in Computer Science}{International Institute of Information Technology}{Hyderabad}{}{}
% \cventry{2015--2019}{B.Tech in Computer Science}{International Institute of Information Technology}{Hyderabad}{}{} %{CGPA -- 8.46}{}
% \cventry{2022--2025 (expected)}{PhD Student in Informatics}{University of Edinburgh}{}{}{Advisor: Tobias Grosser} % arguments 3 to 6 can be left empty
% \cventry{2015--2021}{Bachelor's and MS in Computer Science}{IIIT Hyderabad}{}{}{}
% \cventry{2015--2021}{Dual Degree Bachelor's and MS in Computer Science}{International Institute of Information Technology, Hyderabad}{}{CGPA --- 8.46}{}
% \cventry{2014--2015}{Senior Secondary, CBSE}{Chennai Public School}{Chennai}{}{} %{\textit{Percentage -- 93.2}}{}
% \cventry{2012--2013}{Secondary, CBSE}{Chennai Public School}{Chennai}{}{} %{\textit{CGPA -- 9.8}}{}

%----------------------------------------------------------------------------------------
% EXPERIENCE SECTIONS
%----------------------------------------------------------------------------------------

%Interests/ Keywords/ Summary
% \section{Summary}
% This CV can also be automatically complied and published using GitHub Actions. For details, \href{https://github.com/jitinnair1/autoCV}{click here}.
\vspace{-0.5em}
\section{Professional Summary}
Experience in \textbf{low-level performance engineering, compilers, and algorithms} across the stack:
\begin{itemize}
    \item \underline{implemented and maintain the integer linear programming (ILP) solver in MLIR} \href{https://github.com/llvm/llvm-project/blob/main/mlir/lib/Analysis/Presburger/Simplex.cpp#L1996}{[code]}
    \item optimized an ILP solver for compiler workloads using C++ templates \& AVX-512 vector intrinsics to achieve a \underline{3.6x speedup over SOTA}  \href{https://2021.splashcon.org/details/splash-2021-oopsla/66/FPL-Fast-Presburger-Arithmetic-through-Transprecision}{[Distinguished Paper at OOPSLA]}
    \item research on \underline{cache hierarchy miss rate prediction} \href{https://dl.acm.org/doi/10.1145/3656452}{[paper at PLDI]} % for MLIR Affine
    \item designed \underline{asymptotically better ILP algorithms} for our compiler use-case \href{https://link.springer.com/chapter/10.1007/978-3-031-65627-9_14}{[paper at CAV]}
    \item performance debugging: reading flamegraphs, LLVM IR, x86-64 assembly, architecture diagrams, ...
    \item benchmarking with \href{https://www.amd.com/content/dam/amd/en/documents/epyc-technical-docs/programmer-references/58550-0.01.pdf}{PMCs}, minimizing noise from frequency scaling, SMT, context switches, ...

    % \item benchmarking: minimizing measurement noise from 
    % \item 
    % \item reading LLVM IR or corresponding CFGs to diagnose missed compiler optimizations, and  \\using C++ attributes to guide the compiler to generate the desired code
    % \item using AVX-512 vector intrinsics to vectorize a Simplex-based LP solver
    % \item using C++ template metaprogramming to minimize branching at runtime
    % \item using Linux perf and flame graphs to locate bottlenecks
\end{itemize}
Strong math background evidenced by experience in theoretical research.
% I also have experience in theoretical AI safety.
% I wrote and maintain the ILP solver in MLIR.
% I upstreamed an ILP solver to the LLVM project and remain its lead maintainer, in which capacity \\ I have reviewed 99 pull requests and worked with up to four other contributors at a time. \\
% My PhD thesis will be on compiler algorithms.

\vspace{-0.4em}
\section{Open-Source Projects}
\begin{tabularx}{\linewidth}{ @{}l r@{} }
\textbf{MLIR Presburger Library} & \hfill
\href{https://github.com/llvm/llvm-project/commits/main?author=superty}{[Commits]} 
\href{https://github.com/search?q=repo\%3Allvm\%2Fllvm-project+\%22Reviewed+By\%3A+arjunp\%22\&type=commits\&p=1}{[Reviewing]}
\href{https://github.com/llvm/llvm-project/pulls?q=is\%3Apr+reviewed-by\%3Asuperty+}{[More Reviewing]} 
\href{https://grosser.science/FPL}{[Website]} \\[3.75pt]
\multicolumn{2}{@{}X@{}}{
\begin{itemize}
\vspace{-0.5em}
\item Lead maintainer of the Presburger library of the LLVM/MLIR open-source compiler framework 
\item Reviewed 99 pull requests and worked with a team of up to four other contributors
\item Added support for multiprecision integers to LLVM based on APInt \& optimized it for our workload 
\end{itemize}}
\end{tabularx}
% , and is tailored to the use case of polyhedral compilation.
% \vspace{-2em}

\vspace{-0.4em}
\section{Work Experience}

% \begin{tabularx}{\linewidth}{ @{}l r@{} }
% \textbf{Visiting Research Student} at the \textbf{University of Cambridge} & \hfill May 2024 --- Present \\[3.75pt]
% \multicolumn{2}{@{}X@{}}{
% \vspace{-1em}
% % \begin{itemize}
% % \item Research on an accurate and efficiently computable predictive model for CPU caches
% % \item Empirically investigated what features must be modeled to achieve accuracy in practice
% % \item Designed and implemented a modeling algorithm that is theoretically and practically efficient
% % \end{itemize}
% I did the second half of my PhD in Cambridge, as my advisor moved there.
% \begin{itemize}
% \item Research on efficient and accurate cache performance prediction for neural networks
% \item Designed an algorithm, proved its correctness \& efficiency, and implemented it in C++
% \item Optimized performance hotspots in the implementation using vector intrinsics
% \end{itemize}
% }
% \end{tabularx}

\begin{tabularx}{\linewidth}{ @{}l r@{} }
\textbf{Student Researcher} at \textbf{Google DeepMind (Paris), France} & \hfill November 2023 --- February 2024 \\[3.75pt]
\multicolumn{2}{@{}X@{}}{Worked on static analysis for tiled linear algebraic kernels. I developed asymptotically faster algorithms for ILP for subclasses relevant to compilers, with proofs of efficiency \& correctness. \hfill \href{https://link.springer.com/chapter/10.1007/978-3-031-65627-9_14}{[Paper @ CAV]}
% I showed that the problem is still NP-hard and devised an algorithm that runs in pseudo-linear time in a parameter that is small in practice. Then, we identified a further restricted subclass that usually suffices, for which I devised a polynomial-time algorithm. The work was \href{https://arxiv.org/abs/2405.11244}{published at CAV 2024}.
}
\end{tabularx}

\begin{tabularx}{\linewidth}{ @{}l r@{} }
\textbf{Research Intern} at \textbf{ETH Z\"urich, Switzerland} & \hfill August 2019 --- April 2020 \\[3.75pt]
\multicolumn{2}{@{}X@{}}{
% \vspace{-1em}
Wrote the MLIR Presburger library (see above) and worked on papers published at OOPSLA '20 and '21.
% \begin{itemize}
% Wrote an optimized C++ library for deciding formulas in Presburger arithmetic, of which ILP is a special case. We upstreamed the library to LLVM/MLIR. Our work won a Distinguished Paper Award at OOPSLA 2021. 
% \item Implemented an algorithm for parametric integer linear programming
% \item Optimized for the common case using template specialization and AVX-512 vector intrinsics
% \item Our paper on the library \href{https://2021.splashcon.org/details/splash-2021-oopsla/66/FPL-Fast-Presburger-Arithmetic-through-Transprecision}{won a Distinguished Paper Award at OOPSLA 2021}
% \item I \href{https://grosser.science/FPL/}{upstreamed} most of my code to the LLVM/MLIR project
% \end{itemize}
}
% \vspace{-1em}
\end{tabularx}
% Wrote a fast C++ library for deciding formulas in Presburger arithmetic. Our paper on the library \href{https://2021.splashcon.org/details/splash-2021-oopsla/66/FPL-Fast-Presburger-Arithmetic-through-Transprecision}{won a Distinguished Paper Award at OOPSLA 2021}. I \href{https://grosser.science/FPL/}{upstreamed} most of my code to the LLVM project.

\begin{tabularx}{\linewidth}{ @{}l r@{} }
\textbf{Teaching Assistant} at \textbf{IIIT Hyderabad, India} & \hfill Monsoon 2018, Spring 2019 semesters \\[3.75pt]
\multicolumn{2}{@{}X@{}}{Served as a TA for the cryptography and algorithms courses.} \\
\end{tabularx}

\begin{tabularx}{\linewidth}{ @{}l r@{} }
\textbf{Research Intern} at \textbf{Tata Institute of Fundamental Research, Mumbai, India} & \hfill May --- July 2018 \\[3.75pt]
\multicolumn{2}{@{}X@{}}{Studied approximation algorithms and hardness of approximation (via probabilistically checkable proofs).} \\
\end{tabularx}

\begin{tabularx}{\linewidth}{ @{}l r@{} }
\textbf{Software Engineering Intern} at \textbf{Google Z\"urich, Switzerland} & \hfill May --- August 2017 \\[3.75pt]
\multicolumn{2}{@{}X@{}}{Worked on performance optimizing RPC calls in a C++ microservices framework.} \\
\end{tabularx}

\begin{tabularx}{\linewidth}{ @{}l r@{} }
\textbf{Coach} at the \textbf{Indian IOI Training Camp} & \hfill 2016, 2018 \\[3.75pt]
\multicolumn{2}{@{}X@{}}{Part of the team that took classes, prepared problems and test data, and evaluated students' solutions at the International Olympiad in Informatics Training Camp (IOITC), which is held to select a team to represent India at the International Olympiad in Informatics (IOI).} \\
\end{tabularx}

\section{Education}
\begin{tabularx}{\linewidth}{@{}l X@{}}	
2024 -- present & Visiting PGR Student at the \textbf{University of Cambridge}, UK \normalsize (Advisor: Tobias Grosser) \\
2022 -- present & PhD in Informatics at the \textbf{University of Edinburgh}, UK \normalsize (Advisor: Tobias Grosser) \\
& \emph{Thesis title: Efficient Static Analysis for Neural Networks} \\
& \emph{I did the second half of my PhD in Cambridge, as my advisor moved there.} \\
% 2019 -- 2021 & Master of Science in Computer Science and Engineering at IIIT Hyderabad \\ 
2015 -- 2021 & Bachelor's and MS in Computer Science and Engineering at \textbf{IIIT Hyderabad}, India \\ 
\end{tabularx}

\vspace{-0.3em}
\section{Skills}
\begin{spacing}{1.24}
% Performance engineering, compilers, benchmarking, discrete mathematics, theoretical CS, algorithms. \\
Compilers, performance engineering, benchmarking, theoretical computer science, algorithms. \\
C++, Python, shell scripting (daily use). x86-64 assembly, LLVM IR (mostly reading, some writing). \\
% Familiarity with GPU architecture (programming massively parallel processors book).
% Familiarity with binary exploitation (\href{https://pwn.college/}{pwn.college} course).
\end{spacing}
\vspace{-0.3em}
\section{Achievements}
\textbf{Bronze Medal at the International Olympiad in Informatics (IOI) '15.}
% at the IOI and was \textbf{awarded a Bronze Medal} for my performance. % Was \textbf{one of four students selected nationally} to represent India at the IOI '15.

% \pagebreak 
\textbf {International Collegiate Programming Competition (ICPC) World Finalist '20.}  \\ My team (tesla\_protocol) competed at the World Finals and \href{https://pc2.ecs.baylor.edu/scoreboard/}{placed 46th internationally}. % We qualified for the World Finals by placing \href{https://algo.codemarshal.org/contests/awf-19/standings}\textbf{7th} in the Asia West Continent Final contest. % We also came \href{https://icpc.baylor.edu/regionals/finder/AMP-2019/standings}{\textbf{6th} in the Asia-Amritapuri regionals} and \href{https://icpc.baylor.edu/regionals/finder/Asia-Kanpur-2019/standings}{\textbf{4th} in the Asia-Kanpur regionals}.
% The ICPC is the most prestigious university-level programming competition.

% \pagebreak
% \textbf{Distinguished Paper Award at OOPSLA 2021}. OOPSLA is a premier systems conference. The award was given to 6 out of 205 submitted papers.


% \pagebreak




% The IOI is the most prestigious school-level programming competition.

% \begin{tabularx}{\linewidth}{@{}l X@{}}
% Some Skills &  \normalsize{This, That, Some of this and that etc.}\\
% Some More Skills  &  \normalsize{Also some more of this, Some more that, And some of this and that etc.}\\  
% \end{tabularx}

\section{Work Experience (Theory)}
\begin{tabularx}{\linewidth}{ @{}l r@{} }
\textbf{Visiting Researcher} at the \textbf{Alignment Research Center} & \hfill August 2025 \\[3.75pt]
\multicolumn{2}{@{}X@{}}{
\vspace{-1em}
Worked with a team of math/theoretical CS PhDs. In two weeks, I found a counterexample for a proposed algorithm, showing it yielded poor approximations. The problem had been open for two months.
% AI safety research towards mechanistic anomaly detection. Specifically, did theoretical research on a toy model involving matrix permanent estimation.
}
\end{tabularx}

\begin{tabularx}{\linewidth}{ @{}l r@{} }
\textbf{Research Scholar} at the \textbf{ML Alignment \& Theory Scholars Program} & \hfill February --- May 2024 \\[3.75pt]
\multicolumn{2}{@{}X@{}}{
Worked on Markov decision process theory. 
% AI safety research on Vanessa Kosoy's learning-theoretic agenda. Specifically, I worked on MDP theory.
} % Vanessa Kosoy on the learning-theoretic agenda for AI alignment.
\end{tabularx}

\section{Work Experience (Theory)}

\begin{tabularx}{\linewidth}{ @{}l r@{} }
\textbf{Undergraduate Research} at \textbf{IIIT Hyderabad, India} & \hfill 2017 --- 2018 \\[3.75pt]
\multicolumn{2}{@{}X@{}}{Worked on hypergraph theory and communication complexity.} \\
\end{tabularx}

\begin{tabularx}{\linewidth}{ @{}l r@{} }
\textbf{Research Intern} at the \textbf{Tata Institute of Fundamental Research, Mumbai, India} & \hfill Summer 2018 \\[3.75pt]
\multicolumn{2}{@{}X@{}}{Studied approximation algorithms and hardness of approximation (via probabilistically checkable proofs).} \\
\end{tabularx}
% Studied results in theoretical computer science (approximation algorithms and hardness of approximation) under the guidance of Prof. Prahladh Harsha.

\begin{tabularx}{\linewidth}{ @{}l r@{} }
\textbf{Teaching Assistant} at \textbf{IIIT Hyderabad, India} & \hfill 2018 --- 2019 \\[3.75pt]
\multicolumn{2}{@{}X@{}}{Served as a TA for the cryptography and algorithms courses.} \\
\end{tabularx}

\section{Publications}
\textbf{Strided Difference Bound Matrices.} \\
Arjun Pitchanathan, Albert Cohen, Oleksandr Zinenko, Tobias Grosser. 
\href{https://link.springer.com/chapter/10.1007/978-3-031-65627-9_14}{Published at CAV 2024.}

\textbf{Falcon: A Scalable Analytical Cache Model.} \\
Arjun Pitchanathan, Kunwar Shaanjeet Singh Grover, Tobias Grosser. 
\href{https://dl.acm.org/doi/10.1145/3656452}{Published at PLDI 2024.}

\textbf{FPL: Fast Presburger Arithmetic through Transprecision.} \\
\normalsize Arjun Pitchanathan, Christian Ulmann, Michel Weber, Torsten Hoefler, Tobias Grosser. \\
\href{https://doi.org/10.1145/3485539}{Published at OOPSLA 2021} and won the \textbf{Distinguished Paper Award}.
%(given to 6 out of 71 papers).

\textbf{Fast Linear Programming through Transprecision Computing on Small and Sparse Data.} \\
Tobias Grosser, Theodoros Theodoridis, Maximilian Falkenstein, Arjun Pitchanathan, Michael Kruse, Manuel Rigger, Zhendong Su, and Torsten Hoefler.  \href{https://doi.org/10.1145/3428263}{Published at OOPSLA 2020.}

\vspace{-0.5em}
\section{Other Research}
\textbf{Provably Efficient LRU Cache Modeling.} \\
Arjun Pitchanathan and Tobias Grosser. \emph{Under preparation.}

\textbf{Compositional Polytope MDPs.} \\
Vanessa Kosoy and Arjun Pitchanathan. \emph{Under preparation.}

\textbf{On the Simple Quasi Crossing Number of $K_{11}$.} Arjun Pitchanathan and Saswata Shannigrahi. \\
Symposium on Graph Drawing and Network Visualization (GD), 2019. Poster with \href{https://arxiv.org/abs/1908.07851}{extended abstract}.

\textbf{Improved Encoding and Counting of Uniform Hypertrees.} \\
Arjun Pitchanathan and Saswata Shannigrahi. \href{https://arxiv.org/abs/1711.03335v4}{Manuscript.}

\textbf{Decision-Theoretic Compression and Value-Aware Measures of Information and Noise.} \\
Tushant Jha, Arjun Pitchanathan, and Kannan Srinathan. 
\href{https://github.com/Superty/dtcompress/blob/master/dtcompress_draft.pdf}{Manuscript.}

\section{Talks}
I have given talks at:
\begin{itemize}
\item the OOPSLA 2021 systems conference \href{https://www.youtube.com/watch?v=UibEvel177w}{[online version of the talk]}
\item the Programming Language Design and Implementation (PLDI) conference in 2022 and \href{https://www.youtube.com/watch?v=ALOvruhw8sI}{2024}
\item the Compilers for Machine Learning (C4ML) workshop at CGO 2021
\item the International Workshop on Polyhedral Compilation Techniques at HiPEAC 2022
\item the EuroLLVM Developer’s Meeting in 2022 and 2024
\item the LLVM Developer’s Meeting in 2021, 2022 (MLIR summit), and 2024
\end{itemize}

\vspace{-0.5em}
\section{Professional Service}
\textbf{Reviewer} for \textbf{ACM Transactions on Architecture and Code Optimization}

\textbf{Program Committee Member} for \textbf{IMPACT '23}, \\
the 13th International Workshop on Polyhedral Compilation Techniques.

\textbf{Artifact Evaluation Committee Member} for \textbf{CC '24}, \\
the ACM SIGPLAN 33rd International Conference on Compiler Construction.

% \section{Selected Coursework}
% % \cvitem{Mathematics}{Mathematics II (Group Theory, Linear Algebra) -- A-}
% % \cvitem{}{Mathematics III (Complex Analysis, Probability) -- B}
% % \cvitem{Formal Methods (Automata and Formal Languages)}, Operating Systems, Introduction to Databases, Artificial Intelligence, Statistical Methods in Artificial Intelligence*, Computer Networks}
% % Operating Systems, Artifical Intelligence, Statistical Methods in Artifical Intelligence
% \begin{tabularx}{\linewidth}{ @{}l r@{} }
%  & \hfill Grade (out of 10) \\
% Algorithms & \hfill 10 \\
% Data Structures & \hfill 10 \\
% Graph Theory & \hfill 10 \\
% Computational Complexity and Advanced Algorithms & \hfill 10 \\
% Computational Complexity Theory & \hfill 10 \\
% Quantum Information and Computation & \hfill 10 \\
% Principles of Information Security (Cryptography) & \hfill 10 \\
% % Principles of Programming Languages & \hfill 10 \\
% Compilers & \hfill 10 \\


% \section{Teaching and Clubs}
% \begin{tabularx}{\linewidth}{ @{}l r@{} }
% \textbf{Teaching Assistant} & \hfill Monsoon 2018; Spring 2019 \\[3.75pt]
% \multicolumn{2}{@{}X@{}}{Served as TA for the Principles of Information Security course as well as the Algorithms course. %Responsibilities included setting assignment problems, setting questions for 
% I held tutorial sessions and office hours, set problems, and graded exam papers.} \\
% \end{tabularx}

\section{Undegraduate Activities}
\begin{tabularx}{\linewidth}{ @{}l r@{} }
\textbf{Coordinator} of the \textbf{Competitive Programming Club} & \hfill Monsoon 2017; Spring 2018 \\[3.75pt]
\multicolumn{2}{@{}X@{}}{One of two coordinators for the Competitive Programming Club that year. This involved conducting events and
teaching topics in computer science, especially involving algorithms and data structures.} \\
\end{tabularx}

\begin{tabularx}{\linewidth}{ @{}l r@{} }
\textbf{Student Systems Administrator} & \hfill Monsoon 2017 \\[3.75pt]
\multicolumn{2}{@{}X@{}}{Had the opportunity to get some hands-on experience in system administration by helping out the university's full-time staff.} \\
\end{tabularx}

\begin{tabularx}{\linewidth}{ @{}l r@{} }
\textbf{Copy-editor} at the \textbf{Ping! Student Magazine} & \hfill Monsoon 2017 \\[3.75pt]
\multicolumn{2}{@{}X@{}}{Copy-edited all the articles in that issue, shepherded some articles through the writing process, and co-authored an article titled "Affording Unemployment for All" on the increasing automation of work.} \\
\end{tabularx}

% \textit{Note: Monsoon and Spring are the two semesters in IIIT Hyderabad's academic calendar.}




% \cventry{August 2019 -- April 2020}{Research Intern}{ETH Z\"urich}{}{}{Worked with Dr. Tobias Grosser at the Scalable Parallel Computing Lab on a fast modern C++ library for Presburger arithmetic.}
% \cventry{Summer 2018}{Research Intern}{Tata Institute of Fundamental Research, Mumbai}{}{}{Surveyed results in theoretical computer science (approximation algorithms and hardness of approximation) under the guidance of Prof. Prahladh Harsha.}
% \cventry{Summer 2017}{Software Engineering Intern}{Google Z\"urich}{}{}{Worked on integrating optimizations in Google's C++ RPC framework into an internal microservices framework.}
% \cventry{Monsoon 2019, Spring 2019}{Teaching Assistant}{IIIT Hyderabad}{}{}{Served as TA for the Algorithms (Monsoon 2019) and Principles of Information Security (Spring 2019) courses at IIIT-H. Responsibilities included setting quiz question papers, correcting answer papers, as well as holding tutorial sessions and office hours.}
% \cventry{Spring 2019}{Teaching Assistant, Principles of Information Security}{IIIT Hyderabad}{}{}{Served as TA for the Principles of Information Security course at IIIT-H. Responsibilities included setting quiz question papers, correcting answer papers, as well as holding tutorial sessions and office hours.}
% \cventry{Monsoon 2019}{Teaching Assistant, Algorithms}{IIIT Hyderabad}{}{}{Served as TA for the introductory Algorithms course at IIIT-H. Responsibilities included setting quiz question papers, correcting answer papers, as well as holding tutorial sessions and office hours.}

% \section{Course Projects}
% \textbf{Decaf Compiler.} Implemented a compiler for a toy programming language supporting loops, function calls, and recursion, with an LLVM back-end. Some type checking and run-time checking was supported. \href{https://github.com/Compilers-Monsoon2018-IIITH/decaff-compiler-Superty}{[github]}

% \textbf{Raft.} Implemented the Raft distributed consensus protocol in C++ using gRPC for RPC calls. \href{https://github.com/Superty/raft-grpc}{[github]}.

% \textbf{Shell.} Implemented a basic Bash-like shell with features including piping, I/O redirection, background processes, suspending the active process, and \texttt{fg}. Implemented in C using system calls (\texttt{fork}, \texttt{exec}, etc.)

% \textbf{Ultimate Tic Tac Toe Bot.} Part of a team of two that created a bot that played a 4x4x4x4 version of Ultimate Tic Tac Toe. The bot used the min-max algorithm with iterative deepening and a heuristic. We placed second out of 80+ entries in a tournament conducted among all the students in the course. \href{https://github.com/anurudhp/ai-assigment-1}{[github]}.

% \textbf{OpenGL Bloxorz Clone.} Made a game in OpenGL, based on the popular flash game \textit{Bloxorz}. \href{https://github.com/Superty/bloxorz-clone}{[github]}

% \cvitem{Directory Syncing System}{Implemented a file-sharing system in Python using sockets that syncs directories and uses file hashes for verification.}

%\cvitem{Photo Sharing Website}{Was part of a team of two that made a photo sharing website using web2py where users could upload, like, comment on and search for images. }
%\cvitem{Auction Website}{Was part of a team of four that built an auctioning website. Was responsible for the back-end, which was implemented in express.js, node.js, and MongoDB. }
% \pagebreak
% \begin{tabularx}{\linewidth}{ @{}l r@{} }
% \textbf{Coach} at the \textbf{Indian IOI Training Camp} & \hfill 2016, 2018 \\[3.75pt]
% \multicolumn{2}{@{}X@{}}{Part of the team that took classes, prepared problems and test data, and evaluated students' solutions at the International Olympiad in Informatics Training Camp (IOITC), which is held to select a team to represent India at the International Olympiad in Informatics (IOI).} \\
% \end{tabularx}
% Formal Languages and Automata Theory & 9 \\ \hline
% Mathematics II (Group Theory, Linear Algebra) & 9 \\ \hline
% Artificial Intelligence & 9 \\ \hline
% Statistical Methods in Artificial Intelligence &  \\ \hline

% %----------------------------------------------------------------------------------------
% %	SKILLS
% %----------------------------------------------------------------------------------------
%

% \vfill
% \center{\footnotesize Last updated: \today}
\end{document}

